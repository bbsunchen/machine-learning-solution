\documentclass{amsart}
\usepackage{fullpage}
\usepackage{color}
\newcommand{\set}[1]{\ensuremath{\left\{#1\right\}}}
\newtheorem{quest}{Question}
\begin{document}
\title{Latex Example}
\author{Your Name}
\maketitle

%this is a comment

\section{Question 1}\label{sec:q1}
Each homework answer should be a section. Do not rearrange the order in which you answer homework questions (i.e. answer question 2 first, then question 3, and answer question 1 last). That will just make things confusing.

This is Section \ref{sec:q1}. Note how the source code uses the label and ref commands. Latex is very easy to use once you get used to it. You can either type it in a text file or use an editor.

This is a new paragraph because of the preceding blank line. This text is \textit{italic}, \textbf{bold}, \underline{underlined}. You can have \texttt{monospace text} that is code for code. 
\begin{verbatim}
You can
also have
larger
blocks like that using the verbatim environment
\end{verbatim}

This is a bulleted list:
\begin{itemize}
\item first item
\item second item
\item You must have at least one item or this will not compile
\end{itemize}

This is a numbered list:
\begin{enumerate}
\item first
\item second
\end{enumerate}

\section{Inline Math}\label{sec:inline}
You can have math within text like this: $1^{32}+x_{var}^47 = 3$, we also have symbols $=, \leq, \geq, <, >$. This is a set: $\set{1,2,3}$. Within text, math needs to be enclosed in dollar signs.


This is a vector $\vec{x}$. This is a dot product $\vec{x}\cdot\vec{y}$. This is a fraction $\frac{1}{2}$. We can make other dots: $\dots$ and $\cdots$ and $\vdots$
\section{Separate equations}\label{sec:equation}
We now prove $(1+1)+1=3$. Add two backslashes to start a new line in the equations:
\begin{align*}
(1+1)+1 &= (2) + 1 & \text{(this is known)}\\
        &=2+1 &\text{note how the ampersand symbol aligns things}\\
        &=3
\end{align*}
Check out how text within equations was added in the previous lines.

We can do the same with equation numbering (we use align instead of align*):

\begin{align}
(1+1)+1 &= (2) + 1 & (improperly added text)\label{eqn:thing}\\
        &=2+1 \nonumber\\
        &=3\nonumber
\end{align}
We assigned a number to Equation \ref{eqn:thing}.

\section{Other useful patterns}.
\begin{align*}
x &= \sum\limits_{i=1}^{n} y_{i}\prod\limits_{j=1}^{32} \beta_{j}\\
z &= \begin{cases}
      1 & \text{ if }a<b\\
      2 & \text{ otherwise}
     \end{cases}\\
y &= \arg \min\limits_{\alpha} \frac{1+\alpha}{\alpha + \log(\alpha) + \sin(\alpha)}\\
q &= \left( 3^8 \right) * \left[6_4^8\right] & \text{left and right rescale parentheses}\\
R &= \left(
     \begin{matrix}
     1 & 2 & 3\\
     4 & 5 & 6
     \end{matrix}
     \right)\\
a &\leq \int_{0}^{\infty} \frac{df(x)}{dx} \frac{\partial g(y,z)}{\partial z}~dy\\
s &= \text{spaces in math mode:}|~|\quad|\qquad|.
\end{align*}
\end{document}
